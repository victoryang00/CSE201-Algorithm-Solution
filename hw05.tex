\documentclass[a4paper]{article}
\usepackage{amsmath}
\usepackage{amsthm}
\usepackage[left=1.8cm,right=1.8cm,top=2.2cm,bottom=2.0cm]{geometry}
\usepackage{enumerate}
\usepackage{fancyhdr}
\usepackage{xpatch}
\usepackage{graphicx}
\usepackage{float}
\usepackage{subfigure}
\usepackage{amsfonts}
\usepackage{mathtools}
\usepackage{framed}
\usepackage{multicol}
\usepackage{fontspec}
\usepackage{float}
\usepackage{algpseudocode}
\usepackage{extarrows}
\usepackage{algorithm}
\usepackage{tikz}
\usepackage{caption}
\makeatletter

\printanswers


\AtBeginDocument{\xpatchcmd{\@thm}{\thm@headpunct{.}}{\thm@headpunct{}}{}{}}
\makeatother

\pagestyle{fancy}
\renewcommand{\baselinestretch}{1.15}
\newcommand{\code}[1]{\texttt{#1}}
\usepackage{paralist}
\let\itemize\compactitem
\let\enditemize\endcompactitem
\let\enumerate\compactenum
\let\endenumerate\endcompactenum
\let\description\compactdesc
\let\enddescription\endcompactdesc

% shorten footnote rule
\xpatchcmd\footnoterule
  {.4\columnwidth}
  {1in}
  {}{\fail}

\title{CSE 201: Homework 5}
% \author{\textbf{Yiwei Yang} \\ \texttt{ yyang363@ucsc.edu}}


\begin{document}
\maketitle
\section{Fractional Knapsack problem}
\subsection{Optimal Substructure}
The optimal strategy is to choose the item that has maximum value vs. weight ratial.

\subsection{Pseudocode}
supposet the item is defined as the struct of int value and weight.
\begin{algorithm}
  \caption{FractionalKPGreedy$(item_arr,n,w)$}\label{alg:cap3}
  \begin{algorithmic}[1]
    \State{SORT(A,n, [item a, item b]()\{return a.value/a.weight>b.value/b.weight\})}
    \State{initialize $final\_val$ with $0.0$}
    \For{i in 0..n}
    \If{$item\_arr[i].weight<w$}
    \State{$w-=item\_arr$}
    \State{$final\_val +=item\_arr[i].value$}
    \Else
    \State{$final\_val+=item\_arr[i].value +w/item\_arr[i].weight$}
    \State{\textbf{break}}
    \EndIf
    \EndFor
  \end{algorithmic}
\end{algorithm}
\subsection{Proof of correctness}
Assume there's an instance that's better the definition of optimal substructure which is $\{p_1,...p_n\}$, the result outcome by the above algorithm is $\{q_1,...q_n\}$ w.l.o.g, the items have already sorted by decreasing order of $\frac{value_i}{weight_i}$. We have $\sum_{i=1}^n p_i value_i<\sum_{i=1}^n q_i value_i$. Let $i$ be there first index at which $p_i \neq q_i$. By the design of our algorithm, it must be that $p_i>q_i$. By the optimality, there must exist an item $j>i$ such that $p_j<q_j$. Consider a new solution $q^{\prime}=\left\{q_1^{\prime}, q_2^{\prime}, \ldots, q_n^{\prime}\right\}$ where $q_k^{\prime}=q_k$ for all $k \neq i, j . \quad q^{\prime}$ will take a little more of item $i$ and a little less of item $j$ compared to $O P T: q_i^{\prime}=q_i+\epsilon$, $q_j^{\prime}=q_j-\epsilon \frac{weight_i}{weight_j}$. The total weight does not change: $\sum_{i=1}^n q_i^{\prime} weight_i=\sum_{i=1}^n q_i weight_i$. Yet the total value strictly increases: $\sum_{i=1}^n q_i^{\prime} value_i=\sum_{i=1}^n q_i value_i+\epsilon value_i-\epsilon \frac{weight_i}{weight_j} value_j>\sum_{i=1}^n q_i value_i$. The $q^{\prime}$ is not the optimal solution then the optimal which is a contradiction. So the above solution is optimal.

\section{Depth-First Search over Directed Graph}
\subsection{Give the values of entryTime and finishTime of $dfs(A, 0)$}
The answer is

\begin{center}
  \begin{tabular}{ |c|c|c| }
    \hline
    ID & entryTime & finishTime \\ \hline
    A  & 0         & 9          \\
    B  & 1         & 4          \\
    C  & 5         & 8          \\
    D  & 6         & 7          \\
    E  & 2         & 3          \\
    \hline
  \end{tabular}
\end{center}

\subsection{Use DFS to get loop time}
\begin{algorithm}
  \caption{efficientDFS$(arr,n)$}\label{alg:cap4}
  \begin{algorithmic}[1]
    \State{use nodes to build $graph$}
    \State{dump all nodes into a vector of nodes $vs$}
    \State{randomly find a node $x$ to start}
    \State{$time= dfs(x,0)$}
    \While{$true$}
    \For {v in $0..vs.size()$}
    \State{$time = dfs(vs[v],time)$}
    \State{\textbf{continue}}
    \EndFor
    \If{$v = vs.size()$}
    \State{\textbf{break}}
    \EndIf
    \EndWhile
  \end{algorithmic}
\end{algorithm}

The inner loop's running times are how many strongly connected components starting from the loop point are and each DFS will be a DFS inside the SCC.

For instance if the order of the point is $D$, we first marked $DE$ as visited. then $C$, only $C$ is marked, but $CDE$ are all marked. then if $A$, $AB$ will be marked, but $ABCDE$ are all traversed once. the worst running time is E->D->C->B->A, n+m + n-1+m-(1/2/3) +.. 1+1 + n+n-1+...1 =$O(n^2+m^2)$, the best running time is just one DFS $O(n+m1)$
\section{Dijkstra's Algortihm}
The process is 
\begin{center}
  \begin{tabular}{ |c|c| }
    \hline
    v & d[v] \\ \hline
    B & 2\\
    C & 1\\
    D & 4\\
    \hline
  \end{tabular}
\end{center}
Then update
\begin{center}
  \begin{tabular}{ |c|c| }
    \hline
    v & d[v] \\ \hline
    E & 12 \\
    \hline
  \end{tabular}
\end{center}
Then update
\begin{center}
  \begin{tabular}{ |c|c| }
    \hline
    v & d[v] \\ \hline
    F & 4 \\
    \hline
  \end{tabular}
\end{center}
Then update\begin{center}
  \begin{tabular}{ |c|c| }
    \hline
    v & d[v] \\ \hline
    H & 7 \\
    \hline
  \end{tabular}
\end{center}

Then update\begin{center}
  \begin{tabular}{ |c|c| }
    \hline
    v & d[v] \\ \hline
    G & 8 \\
    \hline
  \end{tabular}
\end{center}

Then update
\begin{center}
  \begin{tabular}{ |c|c| }
    \hline
    v & d[v] \\ \hline
    E & 11 \\
    \hline
  \end{tabular}
\end{center}
\subsection{Parents of each vertex}
\begin{center}
  \begin{tabular}{ |c|c| }
    \hline
    v & parent[v] \\ \hline
    B & A \\
    C & A \\
    D & A \\
    E & G \\
    F & B \\
    G & H \\
    H & F \\
    \hline
  \end{tabular}
\end{center}

\section{Prim's and Kruskal's Algorithm}

\subsection{Proof exist a $BC$ composed tree}

Suppose that $BC$ does not belong to some minimum spanning tree of G.

Then there exists a path from B to C in some other minimum spanning tree, called T.

For any edge (x, y) on the path from A to u, we’ll take the edge (x, y) out of the tree T and add edge (C, B) to tree T. This new tree is to be called T’.

There still exists a path from x to y in T’ through edge (C, B).

Weight of (C, B) $\leq$ Weight of all edges (inclusive of edge (x, y))

Implies

Weight of T’ $\leq$ Weight of T

Therefore T’ is the minimum spanning tree a contradiction!

Therefore $BC$ is an element of some minimum spanning tree.

\subsection{Proof exixst a $AD,DE,FG$ composed tree}
Since $AD,DE,FG$ are the smallest-sized three edge.

The proof is that for sorted edges of G in non-descending order, we would want that among the edges of same weight, the edges which are contained in T are placed in first positions.

Kruskal’s algorithm will return T when run on E if sorted in the above mentioned manner.

T: $e_1$ $\leq$ $e_2$ $\leq$ , … , $\leq$ $e_m$

T’: $e’_1$ $\leq$ $e’_2$ $\leq$ , … , $\leq$ $e’_m$

Weight of $e_i$ = Weight of $e’_i$ where 1 $\leq$ i $\leq$ m

Since the algorithm always places edges of T first, the edges of T will be chosen to T’. That proves we will first come into the least-sized three edge.
\section{Miller Rabin}
\subsection{Pseudocode}
$q$ can be starting from $s=n-1,q=0$ and $\text{while }(s\%2==0),q++,s>>=1;$ to find the $q$. Constantly do $s^a$ operation, if a non-trivial a's root is found, it can be determined that it is not a prime number.

To find $a^q$ faster, we can use following Pseudocode
\begin{algorithm}
  \caption{FastPower$(a,exp)$}\label{alg:cap5}
  \begin{algorithmic}[1]
		\If {$exp = 1$}
			\State{return $a$}
		\EndIf
		\If {$exp = 0$} 
			\State{return $1$}
		\EndIf
		\State{$d = FastPower(a, exp >> 1)$}
		\State{$d = d * d$}
		\If {$(exp & 1) = 1$} 
			\State{$d *= a$}
		\EndIf
		\State{return $1$}
  \end{algorithmic}
\end{algorithm}
\subsection{Probality if 1 witness is $\frac 1 4$ what if k independent witnesses if n is a prime}
The confidence for k independent witnesses if n is composite is $1-\frac {1} {4}^k$ and the confidence for prime is $\frac 1 4^k$
\end{document}
