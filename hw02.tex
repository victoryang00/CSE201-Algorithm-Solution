\documentclass[a4paper]{article}
\usepackage{amsmath}
\usepackage{amsthm}
\usepackage[left=1.8cm,right=1.8cm,top=2.2cm,bottom=2.0cm]{geometry}
\usepackage{enumerate}
\usepackage{fancyhdr}
\usepackage{xpatch}
\usepackage{graphicx}
\usepackage{float}
\usepackage{subfigure}
\usepackage{amsfonts}
\usepackage{mathtools}
\usepackage{framed}
\usepackage{multicol}
\usepackage{fontspec}
\usepackage{float}
\usepackage{algpseudocode}
\usepackage{algorithm}
\usepackage{tikz}
\makeatletter

\printanswers


\AtBeginDocument{\xpatchcmd{\@thm}{\thm@headpunct{.}}{\thm@headpunct{}}{}{}}
\makeatother

\pagestyle{fancy}
\renewcommand{\baselinestretch}{1.15}
\newcommand{\code}[1]{\texttt{#1}}
\usepackage{paralist}
\let\itemize\compactitem
\let\enditemize\endcompactitem
\let\enumerate\compactenum
\let\endenumerate\endcompactenum
\let\description\compactdesc
\let\enddescription\endcompactdesc

% shorten footnote rule
\xpatchcmd\footnoterule
  {.4\columnwidth}
  {1in}
  {}{\fail}

\title{CSE 201: Homework 2}
% \author{\textbf{Yiwei Yang} \\ \texttt{ yyang363@ucsc.edu}}


\begin{document}
\maketitle
\section{Using only accurate statements about Limits, prove that $f(n)+o(f(n))=\theta(f(n))$.}
It means a function add a function that is not greater than itself equal to the same order with itself.     
\begin{proof}
The condition $h(n)=o (f(n))$, we have $lim_{n\to \infty }\frac{h(n)}{f(n)}=0$, so there's $n_0$ that make $\forall n\geq n_0$, we have $h(n)\leq f(n)$ and $f(n)+h(n)\leq 2 f(n)$ and then $f(n)+h(n)=O(f(n))$.

To prove the $\Omega(f(n))$, we use the conclusion of $f(n)=\Omega(g(n)) \Leftrightarrow \liminf _{n \rightarrow+\infty} \frac{|f(n)|}{g(n)}>0$

We have  $\liminf _{n \rightarrow+\infty} \frac{|f(n)+h(n)|}{|f(n)|}=\liminf _{n \rightarrow+\infty}|f(n)| \cdot \frac{\left|1+\frac{|h(n)|}{|f(n)|}\right|}{|f(n)|} \geq 1>0$

We then got $f(n)+o(f(n))=\Omega(f(n))$, so that $f(n)+h(n)=\Theta(f(n))$, where $h(n)=o(f(n))$.
\end{proof}
\section{Given the following algorithm for REVERSE-SORT}
\begin{algorithm}
    \caption{REVERSE-SORT}\label{alg:cap}
    \begin{algorithmic}[1] 
    % \Require $n \geq 0$
    % \Ensure $y = x^n$
    \For{$j = 1$ to $n-1$}
    \State $largest \gets j$
    \For{$i = j+1$ to $n$}
    \If{$A[i]>A[largest]$}
    \State $largest \gets i$
   
    \EndIf
    \EndFor
    \State \textbf{exchange }$A[i]$ with $A[largest]$
    \EndFor
    \end{algorithmic}
    \end{algorithm}
\subsection{Prove that REVERSE-SORT sorts the values in A[1..n] in reverse order when $n \geq 1$.}
\begin{proof}
    First, denote $A[i..j]$ the sublist of A from index i to index j with both side inclusive and loop invariant \code{Inv1} for the outer \textbf{for} statement, \code{Inv2} for the inner \textbf{for} statement.
    
    Before the inner \textbf{for} statement, $A[1..j]$ is a sorted permutation of the original $A[1..j]$. \code{Inv1} holds at the start because $A[2..2]$ is sorted. To prove the \code{Inv1} holds \textbf{for} each iteration. We need to prove that after the \textbf{exchange}, $A[1..j]$ is still a sorted permutation of the original $A[1..j]$. Concerning the inner \textbf{for} loop, \code{Inv2}, after which states \textbf{if}, $A[i..j]$ are each greater than or equal to $A[largest]$. \code{Inv2} is true upon the initialization because of the condition $i=j+1$ and $A[i]>A[largest]$. The inner loop maintains the invariant because it just record the current largest index without hurting the invariant or the data $A$. Then we exchange the data in $A$ the current index with the largest. When the inner loop terminates, we have:
\begin{enumerate}
    \item $A[i..n]$ is sorted and is at most equal to the $A[largest]$. By default is true if $i=n$ and because $A[i..n]$ is sorted and $A[i]<A[largest]$ if $i>0$.
    \item $A[j+1..i]$ is sorted and at least equal to $A[large]$ because the loop invariant holds before $i$ was incremented and that invariant was $A[j..i]\geq A[largest]$
    \item $A[i]=A[i-1]$ if the loop is executed at least once and $A[i-1]=A[largest]$ if the loop did not execute at all.
\end{enumerate}
Because \code{Inv1} is maintained after a loop iteration, it is also maintained even when the outer loop terminates. When the outer loop terminates, $j=n-1, i=n$ and $A[1..n]$ is sorted
\end{proof}
\subsection{What is the time complexity of REVERSE-SORT?}
\begin{proof}
The running time of Insertion Sort can be described by a function $f(N)$ where
$$
f(n)=n \cdot t_1+(n-1) \cdot t_2+x \cdot (t_3+t_4)+(x-1) \cdot t_5+(n-1)\cdot t_8
$$
We do not know $x$ exactly but we can derive an upper bound for $x$. For a fixed $j$, i.e., for one iteration of the outer loop, let $x_j$ be the maximum number of times Statement 4 can be executed. $i$ starts at $j-1$ therefore statement 4 is executed no more than $j-1+1=j$ times because $i$ is decremented by 1 within the while loop. Thus $x_j=j$ and an upper bound on $x$ is $\sum_{j=2}^N x_j=$ $\sum_{j=2}^N j=\left(\sum_{j=1}^N j\right)-1=N(N+1) / 2-1=\left(N^2+N-2\right) / 2$

Using the upper bound on $x$ that we have established (above), we can express $f(N)$ in the worst case as
$$
f(N)=\Theta (n^2)
$$
     
\end{proof}
\section{Prove both of the following by Induction. }
\subsection{ For n$\geq$1, 1×2 + 2×3 + 3×4 + ... + (n)(n+1)  =  (n)(n+1)(n+2)/3}
\begin{proof}
    \begin{enumerate}
        \item Base case: $n=1$ then $1×2 + 2×3 + 3×4 + ... + (1)(2)  =  (1)(2)(3)/3=2$
        \item Assume the statement is true for $n=k$ then $1×2 + 2×3 + 3×4 + ... + (k)(k+1)  =  (k)(k+1)(k+2)/3$
        \item Prove the statement is true for $n=k+1$ then $(1)(2)+(2)(3)+(3)(4)+\cdots+(k)(k+1)+(k+1)(k+2)$=$\frac{(k)(k+1)(k+2)}{3}+(k+1)(k+2)$=$\frac{k}{3}(k+1)(k+2)+(k+1)(k+2)$=$\frac{k+3}{3}(k+1)(k+2)$=$\frac{(k+1)(k+2)(k+3)}{3}$
    \end{enumerate}

\end{proof}
\subsection{For n$\geq$1, the number of subsets of \{1, 2, ..., n\} that have an odd number of elements in them is $2^{n-1}$.}
\begin{proof}
    \begin{enumerate}
        \item Base case: $n=1$ then the number of subsets of \{1\} that have an odd number of elements in them is $2^{1-1}=2^0=1$, $n=2$ then the number of subsets of \{1,2\} that have an odd number of elements in them is $2^{1-0}=2^1=2$
        \item Suppose the statement is true for $n=k$ and $n=k+1, k \geq 3$, w.l.o.g. suppose here $k$ is odd, then the number of subsets of $\{1,2,\cdots,k\}$ that have an odd number of elements in them is $2^{k-1}$ and $2^k$. 
        \item We have for $n=k+2$ which is odd, let $k=2b+1, 1\leq b<\infty, n = 2b+3$ the number of subsets of $\{1,2,\cdots,k+2\}$ is $2^{2b} - \sum^{c=b}_{1}{C^{2c+1}_{k}} + \sum^{c=b+1}_{1}{C^{2c+1}_{k}}=2^{2b}-2^{2b}+2^{2b+2}=2^{2b+2}=2^{k+1}$

        We have for $n=k+3$ which is even, let $k=2b, 2\leq b<\infty, n = 2b+2$ the number of subsets of $\{1,2,\cdots,k+2\}$ is $2^{2b} - \sum^{c=b}_{1}{C^{2c}_{k}} + \sum^{c=b+1}_{1}{C^{2c+2}_{k}}=2^{2b}-2^{2b}+2^{2b+2}=2^{2b+2}=2^{k+2}$
    \end{enumerate}
\end{proof}
\section{This question is about recurrence equation $T(n)=9 T(\lfloor n / 3\rfloor)+5 n$ for $n \geq 1$, where $T(0)=c$.}
\subsection{Fill in the table below by using the recursion tree for this equation.}
\begin{tabular}{|c|c|c|c|}
    \hline Level & # Problems & Size of each Problem & Amount of "Work" at this Level \\
    \hline 0 & 1 & n & 5n \\
    \hline 1 & 9 & $\frac{n}{3}$ & 15n \\
    \hline$\ldots$  &  &  & \\
    \hline $\mathrm{t}$ & $9^t$ & $\frac{n}{3^{t}}$ & $5*3^tn$ \\
    \hline$\cdots$ & & & \\
    \hline Leaf & $9^{ {lg_{3}n} }=n^2$ &$\frac{n}{3^{ {lg_{3}n} }}=1$ &$5*3^{ {lg_{3}n} }n=5n^2$  \\
    \hline
    \end{tabular}
\subsection{Your Recursion Tree answer from 4a) should suggest that T(n) is O($n^2$).}
The recursion tree suggests that 
$$T(n)=\sum^{k-1}_{i=0} 5*3^i n + 9^{k-1}*T(0)=5*\frac{1-3^{k-1}}{1-3} +O(n^2)*T(0)=O(n)+O(n^2)=O(n^2)$$ 

For substitution methods:

Recurrence: $T(n)=9 T(n / 3)+5 n$ for $n \geq 2$ and $T(0)=c$. We'll first use Backwards Induction, being careful about Base Cases, and then Forward Induction to ensure that our proof works.
Backwards Induction: Strong Induction Hypothesis: We want to prove that $T(n) \leq d_1 n^2-d_2 n$ for $n>n_0$, with $d_1, d_2$ and $n_0$ to be determined. (The proof won't work without $d_2$, just as it didn't work for the second Substitution Method example in class.)
We'll use the Strong Induction Hypothesis for $n / 3$ which is less than $n$ (but we have to be careful about Base Cases later!). The Induction Hypothesis applied to $n / 3$ tells us that $T(n / 3) \leq d_1 n^2 / 9-d_2 n / 3$, so:
$$
\begin{aligned}
T(n)=& 9 T(n / 3)+5 n \leq 9 *\left[d_1 *(n / 3)^2-d_2 n / 3\right]+5 n \\
&=d_1 n^2-3 d_2 n+5 n,
\end{aligned}
$$
and for the Induction Step, we need to show this value is $\leq d_1 n^2-d_2 n$ for $n>n_0$.
Cancelling out the $d_1 n^2$ term on both sides, that will be true if $-3 d_2 n+5 n \leq-d_2 n$, which (dividing by $n$ and moving terms around) is algebraically equivalent to saying that $5 \leq 2 \mathrm{~d}_2$, or $2.5 \leq \mathrm{d}_2$.

Well, the easiest way to have this be true is to pick $d_2=2.5$, but bigger values of $d_2$ would also work. To make algebra easier later, let's pick $d_2=3$. And notice that this is independent of $n$, so it seems that it doesn't matter how we pick $n_0$ (or $d_1$ ). But hold on. We need to handle Base Cases.
The Ind Hyp might not be true for $n=0$, since for $n=0$ the Ind Hyp says that $\mathrm{c}=\mathrm{T}(0) \leq 0$, and for positive $\mathrm{c}$, that's not true. So we'll pick $n_0 \geq 1$.
For $n=1$, and $n=2$, we used the Induction Hypothesis, which (since we picked $d_2=3$ ) says that $c=T(0) \leq d_1 n^2 / 9-d_2 n / 3=d_1 n^2 / 9+n$.

For $n=1$,
that says that $T(1)=9 c+5 \leq d_1 / 9-1$, which is true if $d_1 \geq 81 c+54$. For $n=2$,
that says that $T(2)=9 c+10 \leq 4 d_1 / 9-2$, which is true if $d_1 \geq(81 c+108) / 4$.
So as long as we pick a value of $d_1$ that is greater than or equal to the max of those two values, the Base Cases for 1 and 2 work.
Okay, that completes the Backwards Induction. Now let's do the Forward Induction to prove that $T(n) \leq d_1 n^2+d_2 n$,
with $d_1=\max (81 c+54,(81 c+90) / 4), d_2=3$, and $n_0=1$
Base Cases for 1 and 2:
- $\mathrm{T}(1)=9 * \mathrm{~T}(0)+5=9 \mathrm{c}+5$, and that value is $\leq \mathrm{d}_1 * 1-\mathrm{d}_2 * 1$ using the construction we used in the Backward Induction for this Base Case (which we actually should explicitly check to avoid sloppy errors).
- $\mathrm{T}(2)=9 * \mathrm{~T}(0)+10=9 \mathrm{c}+10$ and that value is $\leq \mathrm{d}_1 * 4-\mathrm{d}_2 * 2$ using the construction we used in the Backward Induction for this Base Case (which we actually should explicitly check to avoid sloppy errors).
Strong Induction Hypothesis: $T(n) \leq d_1 n^2-3 n$ for $n \geq 1$
with $d_1=\max \left(9^*(c+1), 9 / 4^*(c+2)\right)$, using $d_2=3$.
Strong Induction Step:
Assume that $T(k) \leq d_1 n^2-3 n$ for all $1 \leq k<n$.
We want to prove that $T(n) \leq d_1 n^2-3 n$ when $n \geq 3$.
(We've already proved this for the Base Cases when $n=1$ and $n=2$.)
By the Recurrence, $T(n)=9 T(n / 3)+5 n$.
Since $n \geq 3,1 \leq n / 3<n$, so we can use the Induction Hypothesis for $k=n / 3$, so:
$$
T(n)=9 T(n / 3)+5 n \leq 9 *\left[d_1(n / 3)^2-3 n / 3\right]+5 n=d_1 n^2-9 n+5 n=d_1 n^2-4 n \text {. }
$$
We want to show that $d_1 n^2-4 n \leq d_1 n^2-3 n$.
Well, that just says that $0 \leq n$, and that's certainly true when $n \geq 2$.
\section{For each recursion below, try to use some version of the Master Theorem to determine a function f(n) such that T(n) = $\Theta$(f(n)).}

\subsection{$T(n)=7 T(n / 2)+n^3$}
$T(n)=\Theta(n^3)$
\subsection{$T(n)=3 T(n / 3)+7 n \lg n$}
$T(n)=\Theta(nlog^2(n))$
\subsection{$T(n)=2 T(n / 4)+5 \sqrt{n}$}
$T(n)=\Theta(\sqrt{\mathrm{n}} \log \mathrm{n})$
\subsection{$T(n)=T(n / 3)+T(2 n / 3)+9 n^2$}
Not applied
\subsection{$T(n)=2 T(n / 2)+n / \lg n$}
$T(n)=0(n \log \log n)$
\subsection{$T(n)=3 T(n / 3)+n \lg n$}
$T(n)=\Theta(n \log ^2 n)$
\end{document}
