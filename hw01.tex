\documentclass[a4paper]{article}
\usepackage{amsmath}
\usepackage{amsthm}
\usepackage[left=1.8cm,right=1.8cm,top=2.2cm,bottom=2.0cm]{geometry}
\usepackage{enumerate}
\usepackage{fancyhdr}
\usepackage{xpatch}
\usepackage{graphicx} 
\usepackage{float} 
\usepackage{subfigure} 
\usepackage{amsfonts}
\usepackage{mathtools}
\usepackage{framed}
\usepackage{multicol}
\usepackage{fontspec}
\usepackage{float}
\usepackage{tikz}
\makeatletter

\printanswers


\AtBeginDocument{\xpatchcmd{\@thm}{\thm@headpunct{.}}{\thm@headpunct{}}{}{}}
\makeatother

\pagestyle{fancy}
\renewcommand{\baselinestretch}{1.15}
\newcommand{\code}[1]{\texttt{#1}}
\usepackage{paralist}
\let\itemize\compactitem
\let\enditemize\endcompactitem
\let\enumerate\compactenum
\let\endenumerate\endcompactenum
\let\description\compactdesc
\let\enddescription\endcompactdesc

% shorten footnote rule
\xpatchcmd\footnoterule
  {.4\columnwidth}
  {1in}
  {}{\fail}

\title{CSE 201: Homework 1}
\author{\textbf{Yiwei Yang} \\ \texttt{ yyang363@ucsc.edu}}


\begin{document}
\maketitle
\section{Prove the $f(n)\in O(g(n))\cap \Omega(g(n))$ iff $\exists c_1>0,\exists c_2>0,\exists n_0>0, \forall n_0$ s.t. $0\leq c_1 g(n)\leq f(n)\leq c_2g(n)$}
\begin{proof}
    \begin{enumurate} 
        \item Suppose we have $f(n)\in O(g(n))\cap \Omega(g(n))$, from $f(n)\in O(g(n))$ we get $\exists c_1>0, \forall n_0 c_1g(n)\leq f(n)$, from $f(n)\in \Omega(g(n))$ we get $\exists c_2>0, \forall n_0 f(n)\leq c_2g(n)$. We proof that $\exists c_1>0,\exists c_2>0,\exists n_0>0, \forall n_0$ s.t. $0\leq c_1 g(n)\leq f(n)\leq c_2g(n)$
        \item Suppose we have $\exists n_1, c_1, \forall n \geq n_1, c_1 g(n) \leq f(n) $ and $\exists n_2, c_2, \forall n \geq n_2, f(n) \leq c_2 g(n)$. Let $n_0 = max(n_1, n_2)$, then we have $\forall n\geq n_0, 0\leq c_1 g(n)\leq f(n)\leq c_2g(n) $ ,$f(n)\in O(g(n))\cap \Omega(g(n))$
    \end{enumurate}
\end{proof}
\section{Find relation of following functions}
\subsection{$f(n)=2^n g(n)={(n lg(n))^2}$}
$f(n)=\omega(g(n))$
 \begin{proof}
 $\lim _{n \rightarrow \infty} \frac{f(n)}{g(n)}= \lim _{n \rightarrow \infty} \frac{2^n}{(n lg(n))^2}$ 
 
 To get the derivative of the function, $\frac{d}{d n}\left(\frac{2^n}{(n \log (n))^2}\right) =\frac{2^n((n lg (2)-2) lg (n)-2)}{n^3 lg ^3(n)}$, Since $\frac {2^n}{n^3 lg ^3(n)}$ is monotonically increasing and bigger than 0 $\forall n > 0$, and $(n lg (2)-2) lg (n)-2$ is monotonically increasing and bigger than 0 $\forall n > 10$, Thus $\lim _{n \rightarrow \infty} \frac{f(n)}{g(n)}=\infty$, and $f(n)=\omega(g(n))$
\end{proof}
\subsection{$f(n)=\sqrt{n} \quad g(n)= lg (n)$}
$f(n)=\omega(g(n))$
\begin{proof}
   To proof $\exists c>0, \forall n>n_0$ s.t. $lg(n)\leq c\sqrt{n}$
   
   $\frac{d}{d n} \frac{\log n}{\sqrt{n}}=\frac{2-\log n}{2 n^{\frac{3} {2}}}$ for $n\geq e^2$, $\frac{f(n)}{g(n)}$ is decreasing, thus $c$ can be $\frac{2}{e}$.
\end{proof}


\end{document}